% ----------------------------------------------------------
% Exemplo de capítulo sem numeração, mas presente no Sumário
\chapter{Introdução}

Grafos são estruturas matemáticas usadas para representar relações entre diferentes dados e são utilizadas para modelar problemas reais em diferentes domínios \cite{graph_applications2020}. Muitas situações do mundo real podem ser convenientemente descritas por meio de um diagrama que consiste em um conjunto de pontos juntamente com linhas que unem certos pares desses pontos. Como exemplo, é possível conectar cidades pela distância, pessoas pelo nível de afinidade, computadores por redes de internet, átomos por ligações químicas, neurônios por sinapses, entre outros. Uma das principais vantagens de se utilizar grafos é a possibilidade de realizar análises tanto discretas quanto topológicas, permitindo modelar dados em uma dimensão adicional.

Em diversos campos de aplicação, a otimização combinatória consiste na busca por soluções ótimas ou aproximadas dentro de um conjunto de soluções possíveis, de maneira a maximizar ou minimizar determinados critérios específicos do domínio. Muitos desses problemas podem ser naturalmente formulados em termos de grafos. Um exemplo clássico é o Problema do Caixeiro Viajante, que busca encontrar a menor rota que parte de uma cidade, visita todas as demais e retorna à sua origem. Tais problemas costumam ser computacionalmente intensivos, exigindo cuidado especial na projeção e implementação de algoritmos eficientes.

O Problema de Localização de Facilidades, conhecido como \textit{Facility Location Problem} (FLP), é um tipo de problema de otimização combinatória que consiste em decidir o melhor local para alocar uma facilidade, de forma a minimizar uma determinada função de custo. O problema foi inicialmente idealizado por \citeonline{weber1909}, consistindo em encontrar pontos em um espaço euclidiano que minimizem a soma dos custos de transporte entre estes pontos e as facilidades existentes.

Ao longo do tempo, surgiram diversas variações do problema, que introduzem restrições ou ampliam o domínio de aplicação. Entre elas, destacam-se as variantes \textit{capacitated}, que consideram que as facilidades possuem um número finito de recursos e podem não conseguir atender a todos os clientes, as variantes hierárquicas, que adicionam uma cadeia de facilidades, por exemplo, uma fábrica que atende uma facilidade intermediária, que por sua vez atende o cliente, e as variantes que impõem restrições de domínio, definindo se o espaço de alocação é discreto ou contínuo. Existem ainda muitos outros tipos que consideram questões específicos do domínio do problema mas todas seguem o princípio básico de alocação de maneira a minimizar uma função objetivo.

Além disso, o problema pode ser estendido para atender situações de decisão multicritério, \textit{Multi-Criteria Decision Making} (MCDM), considerando diversos fatores na alocação, tais como o tempo de deslocamento entre a facilidade e o cliente, o custo de abertura das facilidades, o número de serviços oferecidos, entre outros, podendo ser adaptados conforme o domínio do problema. \citeonline{flp_mcdp_survey} destacam diversas produções acadêmicas que aplicam o FLP em diferentes áreas, como saúde, química, física, economia, psicologia, entre outras, evidenciando o potencial e a versatilidade desta formulação.


Existem diversos tipos de algoritmos e modelos matemáticos que podem ser usados para abordar o FLP, cuja escolha depende tanto da complexidade do domínio quanto da formulação inicial do problema. Um algoritmo que apresenta bom desempenho em um contexto específico nem sempre terá a mesma performance em outro, devido às diferentes restrições, dimensões e critérios presentes em cada domínio. Além disso, problemas multicritério e hierárquicos introduzem ainda mais variáveis e complexidade, tornando difícil prever qual abordagem será mais eficiente.

Sendo assim, torna-se relevante a criação de um ambiente interativo, flexível e genérico, capaz de acomodar diferentes representações do problema, permitir a implementação e comparação de múltiplos algoritmos, e fornecer métricas de desempenho que auxiliem na análise da eficácia e adequação das soluções em distintos cenários. Dessa forma, é possível realizar experimentos em um ambiente controlado, que posteriormente pode ser adaptado a domínios específicos pelo usuário, após a análise da viabilidade, fornecendo uma exploração acadêmica e pedagógica, e contribuindo para a validação de métodos em contextos diversos, promovendo uma compreensão mais profunda das forças e limitações de cada técnica.

Além dos algoritmos tradicionais de otimização, o ambiente proposto permite a aplicação de técnicas avançadas de análise de grafos, que podem ser utilizadas para reduzir a dimensionalidade do problema, extrair métricas topológicas relevantes, como centralidade e intermediação, ou mesmo para gerar agrupamentos de clientes em clusters, facilitando a definição de locais estratégicos para instalação de facilidades. Dessa forma, o usuário pode explorar diferentes representações do problema, combinando múltiplos critérios, estruturas hierárquicas e análises de grafos, tornando a plataforma extremamente versátil e adaptável a múltiplos domínios e contextos.

É importante destacar que embora o foco principal seja o FLP, a plataforma será implementada de forma genérica, permitindo a modelagem e análise de problemas em grafos de maneira geral, abrindo espaço para constantes evoluções nas ferramentas disponíveis e possibilitando aplicações em outros problemas de otimização combinatória que envolvam grafos.

Esta pesquisa busca contribuir para a área de otimização combinatória e modelagem matemática, fornecendo um ambiente genérico o suficiente para poder simular problemas que podem ser representados por grafos, em principal, o problema da alocação das facilidades, de forma a atender diferentes domínios. A implementação permite que o usuário final crie e selecione os critérios, de forma a adequar o ambiente aos requisitos específicos de seu problema, além de fornecer diversas ferramentas para análise de grafos e simulação do problema, com dados e comparações que irão facilitar a otimização. O trabalho proporciona ao aluno o estudo de uma área com um grande número de pesquisas, abrindo espaço para possíveis desenvolvimentos acadêmicos futuros e contribuindo para o desenvolvimento de conceitos matemáticos e computacionais que são muito importantes para enriquecimento acadêmico e oportunidades profissionais.



\section{Objetivos}

\subsection{Objetivo Geral}
Implementar um programa interativo e flexível para o estudo do problema das localização das facilidades, permitindo que o usuário modele diferentes representações do problema, defina critérios personalizados de otimização e aplique múltiplos algoritmos e obtenha resultados que possibilitem a comparação entre técnicas distintas e a análise das métricas geradas.

\subsection{Objetivos Específicos}
Os objetivos de alta prioridade são os que fazem parte da aplicação e de seu funcionamento e são necessários estar prontos até o fim do projeto. Já os de baixa prioridade são objetivos opcionais que agregam no que já está feito, fornecendo funcionalidades extras que não são necessárias para o funcionamento da aplicação.
 
\subsubsection{Objetivos Específicos de Alta Prioridade}
\begin{itemize}
    \item Interatividade: O sistema deve oferecer uma experiência amigável ao usuário, permitindo que ele interaja de forma intuitiva com os componentes do ambiente. Isso inclui seleção de nós, criação de grafos, ajustes de parâmetros e visualização dinâmica de resultados. A interatividade garante a facilidade de uso do problema, permitindo uma experimentação mais dinâmica, e servindo também como uma ferramenta pedagógica;

    \item Abundância de critérios: O usuário deve ter flexibilidade para definir quais critérios deseja otimizar, como distância, custo ou tempo de atendimento, e escolher qual algoritmo será executado entre os disponíveis. Essa liberdade permite explorar cenários variados e comparar resultados, aumentando a capacidade analítica do ambiente;

    \item Mudança de estrutura: O ambiente deve permitir que o usuário selecione o tipo de estruturação do problema a ser tratado, seja hierárquico, multicritério ou padrão. Isso garante que a plataforma seja adaptável a diferentes contextos e níveis de complexidade;

    \item Entrada: O sistema deve possibilitar que o usuário crie grafos manualmente de forma interativa ou importe grafos a partir de arquivos padronizados. Essa funcionalidade facilita a experimentação com dados reais ou simulados;

    \item Saída: O sistema deve fornecer informações claras e de fácil acesso e interpretabilidade sobre os resultados obtidos a partir da simulação, de modo que permita-o a comparar os resultados e métricas de forma facilitada;
    
    \item Persistência do grafo: o usuário pode salvar o estado de seu trabalho atual em um arquivo em seu computador e depois usa-lo como entrada no programa para resumir o uso de onde parou.
    
    \item Otimização: O programa deve entregar resultados próximos do ótimo, aplicando estratégias de modelagem e implementação eficientes. Isso assegura que o ambiente não seja apenas visual, mas também útil para análise de desempenho de algoritmos;

    \item Escalabilidade: O sistema deve lidar tanto com pequenos quanto grandes conjuntos de dados, mantendo desempenho consistente e sem erros. Essa característica é fundamental para que o ambiente possa ser utilizado com dados reais e portado em diferenças contextos independentemente do volume de dados.
\end{itemize}


\subsubsection{Objetivos Específicos de Baixa Prioridade}
\begin{itemize}
    \item Entrada avançado: a existência de outras ferramentas que trabalham em cima de grafos implica em diferentes formas de serializar um grafo. Fazer com que a aplicação aceite arquivos em diferentes formatos;
    
    \item Saída avançado: geração de gráficos e análises automáticas a partir dos resultados obtidos;
    
    \item Persistência do usuário: salvar dados e ações de usuário para garantir recuperação de dados implica na manipulação de dados sensíveis que devem ser lidados com extrema cautela. Implementar gerenciamento de usuários em banco de dados de forma segura;
    
    \item Simulação passo a passo: inicialmente, a aplicação processará os algoritmos de forma automática, sem exibir cada etapa. A implementação de uma funcionalidade que permita ao usuário acompanhar, visualmente, cada passo executado pelo algoritmo na interface é complexa, exigindo camadas adicionais de abstração e controle sobre o fluxo de execução.
\end{itemize}

\section{Estrutura da dissertação}
Está dissertação está organizada em quatro diferentes tópicos. O Capítulo 1, 
de caráter introdutório, apresenta uma contextualização de grafos e sua relação com otimização combinatória,
especificando o problema-alvo e descrevendo oque será de forma breve assim uma justificativa.


O Capítulo 2 irá apresentar um referencial teórico, explorando os fundamentos 
teóricos de teoria de grafos, as várias interpretações do problemas de localização das facilidades, e 
detalhes técnicos sobre a implementação do programa, como os paradigmas de renderização usados,
modelo de arquitetura, entre outros. 

O Capítulo 3 trata dos procedimentos metodológicos que serão utilizados no desenvolvimento do ambiente, 
descrevendo quais serão os algoritmos implementados bem como detalhes da arquitetura e do fluxo de uso da aplicação.

O Capítulo 4 tem como objetivo a discussão dos resultados obtidos, 
focando em mostrar como os objetivos foram alcançados.  

Por fim, a conclusão recapitulada o problema que está
sendo abordado neste trabalho, sintetiza as principais contribuições e apresenta 
possíveis caminhos para pesquisas futuras e extensões do programa.
% \addcontentsline{toc}{chapter}{Introdução}
% ----------------------------------------------------------


