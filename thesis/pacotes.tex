% Aqui você pode adicionar seus pacotes específicos para uso em seu trabalho.
% Em PACOTES PESSOAIS insira os pacotes que desejar.
% ----------------------------------------------------------
% PACOTES BÁSICOS (ESSENCIAIS AO MODELO)
\usepackage{lmodern}			% Usa a fonte Latin Modern
\usepackage[T1]{fontenc}		% Selecao de codigos de fonte.
\usepackage[utf8]{inputenc}		% Codificacao do documento (conversão automática dos acentos)
\usepackage{lastpage}			% Usado pela Ficha catalográfica
\usepackage{indentfirst}		% Indenta o primeiro parágrafo de cada seção.
\usepackage{color,xcolor}		% Controle das cores
\usepackage{graphicx}			% Inclusão de gráficos
\usepackage{microtype} 			% para melhorias de justificação
\usepackage{amssymb}
\usepackage{listings}
% ----------------------------------------------------------
% ----------------------------------------------------------
% PACOTES PARA GLOSSÁRIO
\usepackage[subentrycounter,seeautonumberlist,nonumberlist=true]{glossaries}
% para usar o xindy ao invés do makeindex:
%\usepackage[xindy={language=portuguese},subentrycounter,seeautonumberlist,nonumberlist=true]{glossaries}
% ----------------------------------------------------------
% ----------------------------------------------------------
% PACOTES DE CITAÇÕES (PRINCIPAIS PACOTES DO MODELO)
\usepackage[brazilian]{backref}		% Paginas com as citações na bibl
\usepackage[alf,
	    abnt-repeated-author-omit=yes,
	    abnt-etal-list=0]{abntex2cite}		% Citações padrão ABNT
% É possível utilizar o sistema numérico de chamada, alterando a opção 'alf' para 'num'.
% Outros estilos bibliográficos podem ser usados. Se este for o caso, comente o pacote acima
% e utilize, por exemplo, o comando abaixo
% \bibliographystyle{acm}
% Consulte outros estilos de bibliografia consultando o manual de estilos bibliográficos do
% BibTeX em 'http://www.bibtex.org/'
% ----------------------------------------------------------
% ----------------------------------------------------------
% PACOTES ADICIONAIS (usados apenas no âmbito do Modelo Canônico do abnteX2)
\usepackage{lipsum}			% para geração de dummy text
% ----------------------------------------------------------
% ----------------------------------------------------------
% PACOTES PESSOAIS (USADOS PELO AUTOR -- acrescente aqui seus pacotes)
% ----------------------------------------------------------
\usepackage[portuguese,onelanguage]{algorithm2e}	% para inserir algoritmos (longend,vlined)
% \usepackage{amsbsy}			% para símbolos matemáticos em negrito
% \usepackage{amscd}			% para diagramas
% \usepackage{amsfonts}			% fontes AMS
% \usepackage{amsmath}			% facilidades matemáticas
% \usepackage{amssymb}			% para os símbolos mais antigos
% \usepackage{amstext}			% para fragmentos tipo texto em modo matemático
\usepackage{amsthm}			% para teoremas
\usepackage{hyperref}			% Amplo suporte para hipertexto em LaTeX
\usepackage{cleveref}			% Referência cruzada inteligente
\usepackage{dsfont}			% para o estilo de conjuntos de números $\mathds{R}$
% \usepackage{ifthen}			% comandos de condição em LaTeX
\usepackage{listings}           	% para inserir códigos de outras linguagens de programação
% \usepackage{lscape}             	% para imprimir alguma página no formato paisagem
\usepackage{mathabx}			% conjunto de simbolos matemáticos
% \usepackage{mathrsfs}			% suporte para fontes RSFS
% \usepackage{pdfpages}           	% para inserir páginas PDF no texto

% \usepackage{verbatim}
\usepackage[a4paper,
            left=3cm,
            right=2cm,
            top=3cm,
            bottom=2cm]{geometry}


\usepackage[scaled]{helvet}
\renewcommand{\familydefault}{\sfdefault}

\usepackage{titlesec}
\usepackage{placeins}
% ----------------------------------------------------------
\renewcommand{\ABNTEXchapterfont}{\normalfont\bfseries}
\renewcommand{\ABNTEXsectionfont}{\normalfont\bfseries}
\renewcommand{\ABNTEXsubsectionfont}{\normalfont\bfseries}
