\chapter{Trabalhos relacionados}

A adoção de redes generativas adversariais em conjunto com algoritmos de aprendizado por reforço 
pode trazer benefícios significativos na convergência dos resultados e na geração de conteúdo mais coerente. 
Os autores \citeonline{rl-mario-cgan} propõem um método de PCG baseado em aprendizado por reforço para a criação 
de níveis do jogo Super Mario Bros. Para lidar com o alto grau de complexidade e dimensionalidade dos espaços de 
ação e observação, eles utilizam uma rede generativa adversarial condicional (CGAN) como aproximadora de função. 
Nesse contexto, o agente de RL gera vetores de baixa dimensionalidade, que são então transformados nos padrões 
correspondentes dos níveis por meio da CGAN, permitindo assim um mapeamento mais eficiente e expressivo entre a 
representação vetorial e os elementos estruturais do jogo.

Um dos grandes desafios relacionados a geração procedural de conteúdo em jogos
é a diversidade de conteúdos possíveis, dificultando a generalização de 
algoritmos. Os autores \citeonline{khalifa2025proceduralcontentgenerationbenchmark}
propuseram um \textit{benchmark} para PCG, contendo 12 cenários de jogos diferentes,
testando vários problemas desde criação de fases até definição de conjunto de regras.
Esse \textit{benchmark} é tratado como o primeiro passo para padronizar comparações no campo da geração de conteúdo procedural, 
ajudando pesquisadores a entender as limitações dos algoritmos existentes e desenvolver melhores soluções.
Os autores testaram três abordagens: gerador aleatório, estratégia evolutiva e algoritmo genético, 
analisando quais métodos conseguem resolver melhor os desafios propostos. Eles demonstram 
que alguns problemas são mais difíceis que outros e que a forma de definir os objetivos influencia os resultados obtidos.


Outro desafio relacionado a abordagens de RL que envolvem técnicas de Machine Learning tradicionais é a presença
de dados de treinamento. Os autores \citeonline{awiszus2021worldgangenerativemodelminecraft} propuseram o World-GAN, 
um método inovador para geração de conteúdo procedural via aprendizado de máquina no Minecraft. 
O objetivo foi criar um gerador capaz de produzir mundos tridimensionais com base em apenas um exemplo, 
usando redes adversariais generativas (GANs). Os autores propuseram a criação da tecnologia \textit{block2vec}, 
uma para a representação densa de blocos em jogos baseados em voxel, como Minecraft, inspirado no word2vec, 
permitindo que modelos generativos como o World-GAN lidem com uma ampla variedade de blocos sem 
dependência direta do número total de classes disponíveis.


No contexto de PCG, a variação do tamanho do conjunto de dados inicial é um fator importante a se considerar, 
especialmente quando se deseja treinar geradores de conteúdo sem depender fortemente de grandes datasets. 
\citeonline{Zakaria_2023} propõem uma abordagem inovadora que elimina a necessidade tanto de datasets quanto de 
recompensas moldadas, utilizando uma técnica de aprendizado progressivo com múltiplos tamanhos de níveis. 
Ao começar com níveis pequenos, que fornecem feedback mais denso, é possível facilitar o aprendizado do modelo e 
promover a geração de níveis jogáveis mesmo em tamanhos maiores. Essa estratégia 
se mostra eficaz na criação de fases diversas e controláveis, além de acelerar significativamente o processo de treinamento e geração.


Os autores \citeonline{beukman2022objectivemetricsprocedurallygenerated} 
abordam um dos desafios fundamentais da geração procedural de conteúdo: 
a avaliação objetiva dos níveis gerados. Para isso, propõem duas métricas baseadas em simulação, que analisam o 
comportamento de um agente planejador para medir, de forma independente do jogo, a diversidade e a dificuldade dos níveis. 
A diversidade é quantificada por meio da comparação das trajetórias de ação, utilizando a distância de edição, garantindo 
que diferenças estruturais relevantes sejam capturadas. Já a dificuldade é estimada com base na expansão da árvore de busca 
A* necessária para resolver o nível, refletindo o esforço computacional exigido do agente. Essas métricas superam limitações 
de abordagens anteriores ao focarem em aspectos diretamente relacionados à jogabilidade, em vez de características puramente 
visuais, permitindo uma avaliação mais robusta e generalizável dos conteúdos gerados.


O trabalho de \citeonline{Wang2022TheFF} traz uma nova abordagem na avaliação de conteúdo gerado, propondo uma extensão 
do framework \textit{Experience-Driven Reinforcement Learning} (EDRL), utilizado na geração procedural de níveis de jogos com 
base em aprendizado por reforço. Originalmente eficaz na criação contínua de fases para jogos de plataforma, 
o framework foi aprimorado para considerar múltiplos aspectos da criatividade no design de níveis e da jogabilidade, 
com foco no jogo Super Mario Bros. Os autores modelam a noção de “diversão” como um equilíbrio entre 
divergência estrutural e de gameplay, integrando essas métricas ao sistema de recompensas. 
Além disso, a utilização de um algoritmo baseado em \textit{generative soft actor-critic} em modo episódico 
permitiu acelerar o processo de geração e tornar o sistema mais robusto frente a diferentes estilos 
de jogo e condições iniciais. O estudo evidencia a eficácia do EDRL multifacetado em gerar níveis 
variados, jogáveis e divertidos de forma eficiente.